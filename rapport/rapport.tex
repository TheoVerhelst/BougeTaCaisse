\documentclass{article}

\usepackage[utf8]{inputenc}
\usepackage[T1]{fontenc}
\usepackage[french]{babel}  % pour les règles de français
\usepackage[cm]{fullpage}  % pour utiliser toute la page
\usepackage{times}  % Times New Roman FTW
\usepackage[parfill]{parskip}  % séparer paragraphes par un espace vertical et sans alinéa

\title{Rapport du projet 1 : \textit{Parking Escape}\\cours d'Algorithmique 2 : INFO-F-203}
\author{Verhelst Théo \and Petit Robin}
\date{18 décembre 2015}

\begin{document}

\pagenumbering{Roman}
\maketitle
\tableofcontents
\newpage
\pagenumbering{arabic}

\section{Introduction}
	Ce document est le rapport relatif au projet du cours d'algorithmique 2 (\textit{INFOF-203}) : \textit{Parking Escape}. Nous commencerons par
	introduire le projet avec l'objectif de l'énoncé ainsi que les objectifs visés par l'implémentation. Ensuite seront discutés les choix concernant
	l'implémentation et la modélisation du problème. Pour finir, l'algorithme et l'implémentation seront détaillés avant de conclure.

	\subsection{Résumé de l'énoncé}
		En bref, résumons la consigne de l'énoncé.

		Soit un parking $P$ admettant une et une seule sortie, et représenté par un quadrillage de dimension finie $L\times H$. Soient $(n+1)$ voitures
		$\{v_G, v_1, v_2, \ldots, v_n\}$. Toutes ces voitures occupent un nombre de cases strictement supérieur à 1, et ont une
		orientation qui leur est associée (soit horizontale, soit verticale). L'objectif est d'amener la voiture $v_G$ (appelée voiture \textit{Goal})
		jusqu'à la sortie du parking en respectant les déplacements relatifs à l'orientation de chaque voiture, à savoir : une voiture verticale ne peut
		se déplacer que vers le haut ou vers le bas et une voiture horizontale ne peut se déplacer que vers la gauche ou vers la droite.

		Les informations concernant la disposition du parking pour la résolution sont passées en paramètre au programme à l'aide d'un fichier d'\textit{input}.
		De plus, la sortie attendue du programme doit se faire à la fois sur l'\textit{output} standard et dans deux fichiers d'\textit{output} :

		\begin{itemize}
			\item le premier doit contenir l'entièreté des déplacements effectués dans la solution (toutes voitures confondues)
				  ainsi que la situation finale s'il existe une possibilité, et s'il n'en existe pas, il doit contenir une brève
				  explication de la raison pour laquelle Goal est coincée dans le parking ;
			\item le second doit contenir la situation initiale.
		\end{itemize}

	\subsection{But du projet}
		Les objectifs de ce projet sont à la fois de faire travailler le langage Java vu au cours de Langage de Programmation 2 (INFOF-202) et
		travailler l'implémentation des algorithmes de théorie des graphes vus au cours d'Algorithmique 2. De plus, le travail étant réalisé en binôme,
		un objectif (secondaire) de ce projet est d'entamer le principe le travail de groupe qui sera à appliquer pour le projet d'année.

		Une consigne du projet était de réaliser ce dernier en faisant bon usage des concepts de la programmation orienté-objet, le langage
		obligeant (Java est \textbf{uniquement} OO : tout est objet).

\section{Choix de représentation}
	Les seules consignes concernant l'implémentation étaient de réaliser le programme en Java et en orienté objet. Le choix des classes à implémenter était
	dès lors totalement laissé aux étudiants. Cette section a pour objectif d'expliquer et de détailler les choix faits dans le cadre de notre implémentation.

	Le programme est représenté par un \textit{package} contenant trois classes principales : \textbf{Graph}, \textbf{Situation}, et \textbf{IOManager}
	ainsi que deux classes périphériques : \textbf{Main} et \textbf{SolutionNotFoundException}.

	Ces deux dernières ne servent qu'à :
	
	\begin{itemize}
		\item lancer le programme pour la première (ainsi que maintenir une batterie de tests) ;
		\item et créer une exception définie dans le cadre du package, propre à la résolution pour la seconde.
	\end{itemize}
	
	Il y a dès lors peu de contenu à l'intérieur de celles-ci. Les trois premières quant à elles détiennent le cœur du programme.

	\subsection{Centralisation dans la classe \texttt{Situation}}
		La classe \texttt{Situation} est la classe contenant toute la représentation interne du parking ainsi que tout le traitement relatif aux
		modifications de l'état de ce dernier : mouvements des voitures, gestion des permissions de mouvements, gestion des voitures bloquant
		les mouvements des autres, etc.

		C'est la classe la plus importante du point de vue de la quantité de code car c'est celle qui contient la codification utilisée par les autres.

	\subsection{Séparation des I/O dans une classe \texttt{IOManager}}
		La classe \texttt{IOManager} est une classe contenant exclusivement des méthodes statiques. En effet, le seul but de cette classe
		est de servir d'interface entre le programme chargé de résoudre le problème donné et les fichiers chargés de définir la situation
		initiale ou de décrire la situation finale. Elle offre donc un découplage entre la codification d'entrée de sortie et la résolution
		du problème.

		C'est au sein de cette classe que le parsing du fichier d'input est fait dans le but de créer la situation de départ décrite dans ce fichier.
		C'est également dans cette classe que sont réalisés les formatages pour l'écriture en output lors de l'achèvement du programme.

	\subsection{Classe de résolution algorithmique}
		La dernière classe, à savoir \texttt{Graph} est chargée de générer la solution à partir de la situation de départ fournie en argument
		sur base d'un algorithme décrit dans la section suivante. Le principe général est de bouger une voiture à la fois jusqu'à trouver une
		solution valide. Chaque situation valide est enregistrée et liée à toute celles qui ne diffèrent d'elle que par un mouvement.
		Un graphe des situations se construit alors au fur et à mesure de la recherche de la solution.\\
		Une situation est valide si toutes les voitures sont à l'intérieur du parking, et si aucune voiture n'en chevauche une autre.
		Une solution est une situation où la voiture Goal se trouve à une case de la sortie du parking.

	\subsection{Manipulation d'un \textit{arbre} lors de la génération des solutions}
		Suite à une discussion avec M. Fortz, titulaire du cours INFOF-203, à propos du projet, il nous a été conseillé de ne générer le graphe
		que lorsque c'est nécessaire, donc de ne pas prendre en compte les nœuds ne nous intéressant pas dans la résolution. À savoir : il est
		inutile (\textit{a priori}) de déplacer une voiture placée dans un coin si elle ne gène aucune autre voiture dans l'immédiat.

		La structure de données que nous manipulons dans ce programme est un arbre qui est parcouru par niveaux (parcours en largeur d'un graphe)
		comme expliqué dans la section suivante.

\section{Algorithme}
		On définit, à partir d'une situation donnée, un ensemble de mouvements \textit{intéressants} qui mènent
		à une solution rapidement, c'est-à-dire en un nombre de mouvement plus petit qu'en testant l'ensemble des mouvements valides à
		partir de cette situation. Cela implique que le graphe des situations est dirigé, car si un mouvement donné est intéressant,
		le mouvement inverse ne l'est pas forcément. \\
		Le graphe généré lors de la résolution est défini comme suit :

		\begin{itemize}
			\item la racine du graphe est la situation initiale ;
			\item les fils d'un nœud donné sont tous les nœuds qui ne diffèrent du père que par un unique mouvement intéressant.
		\end{itemize}

	\subsection{Choix de l'algorithme}
		Le choix des mouvements intéressants est une méthode heuristique, dans le sens que ces choix ne mèneront pas de manière certaine
		à une solution optimale (avec un nombre de mouvement minimal), mais la recherche de cette solution est bien plus rapide.
		Pour donner un ordre d'idée, le parcourt exhaustif des situations possibles trouve une solution optimale en à peu près 4000 essais,
		et notre implémentation heuristique trouve une solution en à peu près 100 essais. Un essai corresponds à une situation construite,
		et la situation testée est celle décrite dans l'énoncé.

		L'algorithme parcourt le graphe défini ci-dessus par niveaux, jusqu'à trouver une situation qui est une solution.
		Le parcours par niveau revient à construire l'arbre sous-tendant minimal partant de la situation initiale,
		permettant de trouver le plus court chemin menant à une solution.

	\subsection{Explications}
		Pour éviter de générer tout le graphe, les situations sont construites et intégrées au graphe pendant l'exécution du parcours. \\
		Ceci est réalisé au moyen d'une méthode qui donne, pour une situation donnée, les mouvement intéressants.
		L'algorithme part de la situation de départ, et génère une situation pour chacun des mouvements intéressants.
		Pour chacun des fils ainsi générés, il génère ensuite une situation pour chacun des mouvements intéressants partants de ce fils, etc.

		Cet algorithme est conceptuellement récursif, mais pour des raisons de performances il est implémenté de manière itérative.
\end{document}
