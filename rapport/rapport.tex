\documentclass{article}

\usepackage[utf8]{inputenc}
\usepackage[T1]{fontenc}
\usepackage[french]{babel}  % pour les règles de français
\usepackage[cm]{fullpage}  % pour utiliser toute la page
\usepackage{times}  % Times New Roman FTW
\usepackage[parfill]{parskip}  % séparer paragraphes par un espace vertical et sans alinéa

\title{Rapport du projet 1 : \textit{Parking Escape}\\cours d'Algorithmique 2 : INFO-F-203}
\author{Verhelst Théo \and Petit Robin}

\begin{document}

\pagenumbering{Roman}
\maketitle
\tableofcontents
\newpage
\pagenumbering{arabic}

\section{Introduction}
    Ce document est le rapport relatif au projet du cours d'algorithmique 2 (\textit{INFOF-203}) : \textit{Parking Escape}. Nous commencerons par
    introduire le projet avec l'objectif de l'énoncé ainsi que les objectifs visés par l'implémentation. Ensuite seront discutés les choix concernant
    l'implémentation et la modélisation du problème. Pour finir, l'algorithme et l'implémentation seront détaillés avant de conclure.

    \subsection{Résumé de l'énoncé}
        En bref, résumons la consigne de l'énoncé.

        Soit un parking $P$ admettant une et une seule sortie. Soient $(n+1)$ voitures $\{v_G, v_1, v_2, \ldots, v_n\}$. Toutes ces voitures ont une
        orientation qui leur est associée (soit horizontale, soit verticale). L'objectif est d'amener la voiture $v_G$ (appelée voiture \textit{Goal})
        jusqu'à la sortie du parking en respectant les déplacements relatifs à l'orientation de chaque voiture, à savoir : une voiture verticale ne peut
        se déplacer que vers le haut ou vers le bas et une voiture horizontale ne peut se déplacer que vers la gauche ou vers la droite.

        Les informations concernant la disposition du parking pour la résolution sont passées en paramètre au programme à l'aide d'un fichier d'\textit{input}.
        De plus, la sortie attendue du programme doit se faire à la fois sur l'\textit{output} standard et dans un fichier d'\textit{output} dans le cas où
        aucune solution n'est trouvée (en expliquant brièvement pourquoi aucune solution n'est accessible).

    \subsection{But du projet}
        Les objectifs de ce projet sont à la fois de faire travailler le langage Java vu au cours de Langage de Programmation 2 (INFOF-202) et
        travailler l'implémentation des algorithmes de théorie des graphe vus au cours d'Algorithmique 2. De plus, le travail étant réalisé en binôme,
        un objectif (secondaire) de ce projet est d'entamer le principe le travail de groupe qui sera à appliquer pour le projet d'année.

        Une consigne du projet était de réaliser ce dernier en faisant bon usage des concepts de la programmation orienté-objet, le langage
        obligeant (Java est \textbf{uniquement} OO : tout est objet).

\section{Choix de représentation}
    Les seules consignes concernant l'implémentation étaient de réaliser le programme en Java et en orienté objet. Le choix des classes à implémenter était
    dès lors totalement laissé aux étudiants. Cette section a pour objectif d'expliquer et de détailler les choix faits dans le cadre de notre implémentation.

    Le programme est représenté par un \textit{package} contenant trois classes principales : \textbf{Graph}, \textbf{Situation}, et \textbf{IOManager}
    ainsi que deux classes périphériques : \textbf{Main} et \textbf{SolutionNotFoundException}.

    Ces deux dernières ne servent qu'à lancer le programme pour le premier (ainsi que maintenir une batterie de tests) et la seconde est une exception
    définie dans le cadre du package. Il y a dès lors peu de contenu à l'intérieur de celles-ci. Les trois premières quant à elles détiennent le cœur du
    programme.

    \subsection{Centralisation dans la classe \texttt{Situation}}
        La classe \texttt{Situation} est la classe contenant toute la représentation interne du parking ainsi que tout le traitement relatif aux
        modifications de l'état de ce dernier : mouvements des voitures, gestion des permissions de mouvements, gestion des voitures bloquant
        les mouvements des autres, etc.

        C'est la classe la plus importante du point de vue des proportions car c'est celle qui contient la codification utilisée par les autres.

    \subsection{Manipulation d'un \textit{arbre} lors de la génération des solutions}

\section{Algorithme}
    \subsection{Choix de l'algorithme}
    \subsection{Explications}

\end{document}
