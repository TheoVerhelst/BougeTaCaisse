\documentclass{article}

\usepackage[utf8]{inputenc}
\usepackage[T1]{fontenc}
\usepackage[french]{babel}  % pour les règles de français
\usepackage[cm]{fullpage}  % pour utiliser toute la page
\usepackage{times}  % Times New Roman FTW
\usepackage[parfill]{parskip}  % séparer paragraphes par un espace vertical et sans alinéa

\title{Rapport du projet 1 : \textit{Parking Escape}\\cours d'Algorithmique 2 : INFO-F-203}
\author{Verhelst Théo \and Petit Robin}

\begin{document}

\pagenumbering{Roman}
\maketitle
\tableofcontents
\newpage
\pagenumbering{arabic}

\section{Introduction}
    Ce document est le rapport relatif au projet du cours d'algorithmique 2 (\textit{INFOF-203}) : \textit{Parking Escape}. Nous commencerons par introduire le projet
    avec l'objectif de l'énoncé ainsi que les objectifs visés par l'implémentation. Ensuite seront discutés les choix concernant l'implémentation et la
    modélisation du problème. Pour finir, l'algorithme et l'implémentation seront détaillés avant de conclure.

    \subsection{Résumé de l'énoncé}
        En bref, résumons la consigne de l'énoncé.

        Soit un parking $P$ admettant une et une seule sortie. Soient $(n+1)$ voitures $\{v_G, v_1, v_2, \ldots, v_n\}$. Toutes ces voitures ont une orientation qui leur est
        associée (soit horizontale, soit verticale). L'objectif est d'amener la voiture $v_G$ (appelée voiture \textit{Goal}) jusqu'à la sortie du parking en respectant les
        déplacements relatifs à l'orientation de chaque voiture, à savoir : une voiture verticale ne peut se déplacer que vers le haut ou vers le bas et une voiture
        horizontale ne peut se déplacer que vers la gauche ou vers la droite.

        Les informations concernant la disposition du parking pour la résolution sont passées en paramètre au programme à l'aide d'un fichier d'\textit{input}.
        De plus, la sortie attendue du programme doit se faire à la fois sur l'\textit{output} standard et dans un fichier d'\textit{output} dans le cas où
        aucune solution n'est trouvée (en expliquant brièvement pourquoi aucune solution n'est accessible).

    \subsection{But du projet}

\section{Choix de représentation}
    \subsection{Centralisation dans la classe \texttt{Situation}}
    \subsection{Manipulation d'un \textit{arbre} lors de la génération des solutions}

\section{Algorithme}
    \subsection{Choix de l'algorithme}
    \subsection{Explications}

\end{document}
